\chapter{Introdução}

Este texto é baseado principalmente no livro \href{http://savannah.nongnu.org/projects/pgubook}{Programming from the Ground Up} de Jonathan Bartlett \cite{Bartlett}. A leitura do livro é bastante recomendada. Além uma linguagem simples e fácil de ler os assuntos são abordados mais profundamente do que serão aqui. Ele é distribuído sob a licença \href{http://www.gnu.org/licenses/fdl.html}{Gnu FDL} \cite{Gfdl} e pode ser baixado livremente através deste \href{http://savannah.nongnu.org/projects/pgubook}{link}.

O livro disponibiliza exercícios, alguns dos quais serão reproduzidos aqui. Há também uma lista de emails disponibilizada pelo autor do livro que pode ser acessada clicando \href{http://mail.nongnu.org/mailman/listinfo/pgubook-readers}{aqui}.

Os créditos pelo conteúdo são inteiramente do autor do livro original, enquanto qualquer erro certamente é minha culpa \footnote{Eu ando torcando letras frequentemnete :)}.

\section{Ferramentas}

Este texto trata de programação em Assembly para x86 em ambiente Linux e será feito com base em um \href{http://www.ubuntu.com}{Ubuntu 10.04}. Na medida do possível tentarei apresentar um breve sumário dos comandos utilizados ou links para a documentação.

Os pacotes de desenvolvimento podem ser instalados no Ubuntu a partir do metapacote \textbf{build-essentials} que vai automaticamente instalar o gcc, bibliotecas, cabeçalhos, etc. Recomendo também a instalação dos pacotes \textbf{manpages-dev} e \textbf{manpages-posix-dev} que são pacotes de documentação muito úteis.

\subsection{GCC}

GCC significa Gnu Compiler Collection e é uma coleção de compiladores bibliotecas e ferramentas para várias linguagens \cite{Gcc}.

\subsection{Linux}

Linux é o nome do \emph{Kernel} de uma distribuição \href{http://en.wikipedia.org/wiki/GNU/Linux_naming_controversy}{GNU/Linux}. Ele é o núcleo do sistema operacional, responsável por disponibilizar uma interface homem-máquina, gerenciar os recursos do hardware e prover uma API unificada de programação \footnote{Eu não resisti ;)}, facilitando o acesso pelo usuário e protegendo-o dele mesmo. Entretanto o kernel por si próprio não faz muita coisa, ele precisa de um conjunto de ferramentas para ter uso prático.

Combinando o kernel (Linux) com as ferramentas de usuário do \href{http://gnu.org}{Projeto GNU}, temos um sistema operacional completo \footnote{Diga-se de passagem, o GNU/Linux deve o que ele é hoje a um conjunto enorme de ferramentas criadas por várias pessoas e empresas ao redor do mundo, principalmente pelo projeto GNU, e alguns autores afirmam que chamar toda uma distribuição somente de Linux, seria não dar crédito ao projeto GNU.}.

\subsection{Linguagens de Programação}

Na maior parte do texto usaremos linguagem de baixo nível, Assembly, existindo basicamente três tipos de linguagem:

\begin{description}

\item[Linguagem de Máquina]
É a linguagem que o computador entende e manipula, tratando-se de uma sequência de comandos binários essencialmente ilegíveis para humanos normais.

\item[Linguagem Assembly]\footnote{\textbf{Assembly} é o nome da linguagem, enquanto \textbf{assembler} é o nome do programa que realiza a montagem. Você programa em Assembly e monta com o assembler.}
É essencialmente o mesmo que a linguagem de máquina, exceto que os comandos binários são substituídos por mnemônicos \footnote{Do Grego \textgreek{μνημονικός mnēmonikós} (``da memória''), é uma técnica de memorização que utiliza siglas para lembrar de números por exemplo.} mais fáceis de serem memorizados. Outras coisas são adicionadas para tornar a programação mais fácil \footnote{Mais fácil do que escrever em binário, e só!}.

\item[Linguagem de Alto Nível]
Uma linguagem de alto nível existe para tornar a programação fácil. Quando você programa em Assembly, você precisa lidar com a máqina você mesmo, enquanto que ao utilizar uma linguagem de alto nível, você pode descrever o seu algoritmo através de expressões mais naturais. Em geral, uma simples instrução em linguagem de alto nível é equivalente a várias instruções em Assembly \footnote{Isso explica porque instruções em C ou Java podem ser interrompidas ``no meio'' de sua execução.}.

\end{description}

O texto abordará principalmente linguagem assembly e eventualmente alguns aspectos de linguagens de alto nível. Entendendo bem os conceitos básicos de programação de baixo nível, você estará apto a entender melhor a programação de alto nível.



