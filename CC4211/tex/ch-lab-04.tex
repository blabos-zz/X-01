\chapter{Lab 04 - Bootsector}

Quando o computador inicializa, ele começa a executar instruções eu uma área pré-determnada de memória, usualmente onde começa a BIOS. Então são feitas algumas checagens com o o POST (power-on self test) e finalmente a bios procura o setor de boot de algum dispositivo presente. Se encontrado, esse código é executado e toma controle da máquina, geralmente fazendo a carga do SO.

Em PCs, o setor de boot é o primeiro setor de dados do dispositivo, tendo apenas 512 bytes de tamanho. Essa área ainda é subdividida em alguns pedaços, como a marcação de partições, assinatura do disco, entre outros, restando somente 440 bytes para código executável.

Os últimos dois bytes (0x55 e 0xaa)são a assinatura da MBR.

\section{Objetivos}

\section{Requisitos}

\section{Implementação}

\begin{espacosimples}
\begin{verbatim}
//--------------------------------------------------------------------
// boot.s
.code16
.section .text
.globl _start
_start:
    mov $0xb800, %ax
    mov %ax, %ds
    movb $'X', 0
    movb $0x1e, 1
idle:
    jmp idle
//--------------------------------------------------------------------
\end{verbatim}
\end{espacosimples}

\begin{espacosimples}
\begin{verbatim}
//--------------------------------------------------------------------
as -o char.o char.s
ld --oformat binary -o char char.o

dd if=char of=/dev/sdb
echo -ne "\x55\xaa" | dd seek=510 bs=1 of=/dev/sdb
//--------------------------------------------------------------------
\end{verbatim}
\end{espacosimples}

\section{Conclusão}
