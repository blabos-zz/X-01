\chapter{Introdução}

Nos primórdios da computação os computadores eram manipulados apenas por um
pequeno grupo de técnicos altamente especializados, que precisavam plugar uma
grande quantidade de cabos nos enormes painéis que compunham os computadores da
época, num processo lento e muito suscetível a erros.

Com o desenvolvimento da tecnologia e o advento das linguagens e técnicas de programação
modernas, a tarefa de criar programas de computador pode ser desempenhada por um número maior
de pessoas, muitas vezes, cada uma sendo responsável por apenas uma fração de todo o sistema.
Unir todas essas partes e garantir que elas funcionem juntas não é uma tarefa simples.
Analogamente a correção manual de erros em grandes sistemas de computação é frequentemente
uma tarefa tão dispendiosa quanto criá-lo e mantê-lo funcionando.

Para resolver esses problemas de forma automatizada, com o passar dos anos algumas
ferramentas foram sendo desenvolvidas, como os programas diff e patch que facilitam processo de
aplicar uma correção em uma determinada parte do software, assim como os Sistemas de Controle
de Versão dos quais podemos citar o CVS, o Subversion e Git,
que são utilizados para rastrear e
controlar a evolução do código fonte, o que torna o trabalho das equipes de desenvolvimento um
pouco menos complicado.

Entretanto, a dinâmica atual do mercado de tecnologia gera demandas cada vez mais
agressivas, fazendo com que o desenvolvimento de software seja feito quase que 24 horas por dia,
muitas vezes por equipes geograficamente distantes ou em fuso-horários muito diferentes,
ressurgindo o desafio de integrar vários componentes de software agora produzidos não somente por
pessoas distintas mas por empresas e instituições das mais diversas naturezas e culturas, com o
máximo de qualidade possível minimizando o tempo de desenvolvimento e os custos.

Com projetos compostos por milhões de linhas de código, torna-se impraticável que a
integração entre as partes seja realizada manualmente, pois basta somente um equívoco para
provocar erros muito difíceis de corrigir.

Para tentar resolver esse e outros problemas novos Sistemas de Controle de Versão foram
criados, melhorando os mecanismos de mesclagem de código e buscando uma arquitetura
descentralizada, de forma que as equipes possam trabalhar com um mínimo de interferência entre si,
enquanto a etapa de integração é feita cada vez mais de forma automática.

\section{Objetivos}

O objetivo principal deste trabalho é demonstrar como os Sistemas de Controle de Versão
modernos podem ser utilizados para facilitar o desenvolvimento de software colaborativo,
explorando duas de suas principais características, a arquitetura descentralizada e a facilidade com
que integram peças de código.

Será feito um breve levantamento histórico contextualizando alguns dos principais Sistemas de
Controle de Versão em uso atualmente.

Em seguida demostrar-se-á como a arquitetura descentralizada facilita o desenvolvimento de
ramificações independentes do código fonte com um mínimo de conflitos entre elas, confrontando
esse cenário com situações onde não há ferramenta de controle de versão ou quando a ferramenta
utilizada não permite essa abordagem.

Também será mostrado como as novas técnicas de rastreamento da genealogia de um projeto de
software permitem uma integração muito mais suave e automática do que quando são utilizadas
ferramentas menos sofisticadas ou nenhuma ferramenta.