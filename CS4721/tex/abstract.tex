\begin{resumo}
De modo a atender às diversas demandas do mercado, é cada vez mais comum que
o desenvolvimento de software seja realizado 24 horas por dia, por equipes
distintas trabalhando de forma descentralizada ao redor do globo. Por isso,
são cada vez mais necessárias ferramentas que propiciem o rastreamento e a
integração do código fonte. Alguns sistemas de cotrole de versão que foram
desenvolvidos para auxiliar na primeira tarefa, evoluíram, tornando-se também
boas ferramentas de programação integrado-colaborativa. O foco deste trabalho
é extensão do uso dessas ferramentas para auxiliar o desenvolvimento colaborativo
entre equipes geograficamente distantes.

Palavras-chave: Controle de Versão, Colaboração, Integração, Desenvolvimento de Software
\end{resumo}

\begin{abstract}
In order to meet various market demands, it is increasingly common for software
development is carried out 24 hours a day, by separate teams working in a
decentralized way around the globe. Therefore, they are increasingly necessary
tools that provide tracking and integration of source code. Some version control
systems that were developed to assist in the first task, evolved, becoming too
well-integrated programming tools collaboratively. The focus of this paper is
to extend the use of these tools to support collaborative development among
geographically dispersed teams.

Palavras-chave: Version Control, Colaboration, Integration, Software Development
\end{abstract}