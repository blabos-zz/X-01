\chapter{Histórico}

Neste capítulo será abordado brevemente um pouco da história do versionamento de arquivos
no contexto do desenvolvimento de software, quais foram as necessidades de cada época e quais
as ferramentas desenvolvidas para supri-las, mostrando como foi a evolução deste processo
ao longo dos anos.

\section{A Aurora das Linguagens de Programação Modernas}

No século XIX teares programáveis e tocadores automáticos de piano já implementavam o que
hoje é conhecido por Linguagem de Domínio Específico (DSN - Domain Specifc Language).

No início do século XX  o formalismo dos trabalhos de Alonzo Church -- Cálculo Lambda -- \cite{Church} e
de Alan Turing -- Máquina de Turing -- \cite{Turing} foreneceram as bases matemáticas necessárias para se
expressar Algoritmos.
 
Entretanto, as primeiras linguagens de programação utilizadas por computadores digitais
só foram criadas a partir da década de 1940. O termo portabilidade apareceu na década de
1950 e a partir da década de 1960 começaram a ser desenvolvidos os principais paradigmas
de programação que conhecemos hoje.

De lá para cá a popularização do uso de computadores fez com que a demanda por software
crescesse absurdamente, criando toda uma indústira que movimenta bilões de dólares anualmente.

Acompanhado essas cifras, bilhões de linhas de código tem de ser produzidas, mantidas e atualizadas constantemente, o que nunca foi uma tarefa simples e fica cada vez mais complexa a medida
que os sistemas crescem.

\section{As ferramentas diff e patch}

Até a década de 1970, qualquer correção feita em um programa era feita em
uma cópia do arquivofonte original (se disponível) e encaminhada ao auto
do software para que ele analisasse as alterações e incorporasse no
conjunto oficial de fontes.

A tarefa de procurar alterações dentro dos enormes arquivos de código fonte
era então realizada manualmente pelo programador, linha a linha, num processo
bastante lento.

Em 1974 a primeira versão do programa diff foi liberada juntamente com a quinta
versão do sistema operacional Unix, produzido no Bell Laboratories, por Douglas
McIlroy, basead em um protótipo escrito por James W. Hunt da Universidade de
Stanford. Em 1976 eles publicaram um artigo sobre o algoritmo utilizado \cite{Hunt}.

O programa diff aceita como entrada arquivos ou diretórios e gera como saída
apenas as diferenças entre eles especificando os arquivos e números de linhas
onde elas ocorrem. O formato da saída ficou conhecido pelo nome diff, e
posteriormente patch.

O trabalho de encontrar as diferenças entre os arquivos foi simplificado, mas
a tarefa de aplicar as correções ainda era realizada manualmente.

Para resolver esse problema, em maio de 1985 Larry Wall criou o programa patch \cite{Wall}
\footnote{Larry Wall escreveu a versão original do \emph{patch}. Paul Eggert removeu os limites arbitrários do patch; adicionou suporte a arquivos binários, datação e deleção dos arquivos; e deixou-o em conformidade com o padrão POSIX. Outros contribuidores incluem Wayne  Davison, que adicionaou suporte ao formato unidiff, e David MacKenzie, que adicionaou configurações e suporte a backup. Andreas Grünbacher adicionou suporte a mesclagem (merging).\cite{Man}}.
Esse programa pega um arquivo no formato gerado pelo diff e aplica no arquivo
completo de fontes, automatizando a tarefa de correção. Logo esse processo
passou a ser conhecido como patching, onde criar um patch é gerar um arquivo
no formato diff e aplicar o petch é usar o programa patch para implantar as
alterações.

Esses dois prgramas facilitaram em muito o trabalho de programadores que
precisavam gerenciar várias atualizações e correções, entretanto, uma vez
aplicado o patch, removê-lo não é tarfa simples. Isso podia acontecer quando
por exemplo, um arquivo de diff ou de fonte errados eram utilizados, ou
quando a aplicação de um patch corrigia uma falha, mas causava outras.

O aumento do número de patches trouxe outra demanda, a necessidade de
saber quem aplicou qual patch e quando, bem como desfazer as alterações
caso necessário.


\section{O Nascimento do Controle de Versões}

Uma das mais utilizadas ferramentas de controle de revisão dessa época
foi o RCS, criado em 1982 por Walter F. Tichy, sendo uma das primeiras
a automatizar tarefas de armazenar, recuperar, identificar e mesclar
revisões.

Embora razoavelemente útil para lidar com alguns arquivos de texto
o RCS ainda não supria várias necessidades no desenvolvimento de software.

Como um conjunto de scripts trabalhando em conjunto com o RCS, surge em
julho de 1986 o CVS, criado por Dick Grune. Em 1989, Brian Berliner o
reescreveu completamente em C e Jeff Polk adicionou algumas
funcionalidades posteriormente \cite{Bar}.

Mesmo sendo uma ferramenta bastante sofisticada para a época, o CVS ainda
mantinha os principais problemas do RCS, como só poder
ser utilizado localmente, o que só foi resolvido na década de 1990 por
Jim Kingdon.

\begin{citacao}
I created CVS to be able to cooperate with my students, Erik Baalbergen and Maarten Waage, on the ACK (Amsterdam Compiler Kit) C compiler. The three of us had vastly different schedules (one student was a steady 9-5 worker, the other was irregular, and I could work on the project only in the evenings). Their project ran from July 1984 to August 1985. CVS was initially called cmt, for the obvious reason that it allowed us to commit versions independently.\\

\cite{Grune}
\end{citacao}

O CVS foi construído para auxiliar equipes nas quais os programadores tem
de trabalhar em horários diferentes, mas um por vez. Quando surge a necessidade
de vários programadores trabalharem simultaneamente no mesmo projeto, o design
do CVS já não atende tão bem.

Em 1995 Karl Fogel e Jim Blandy criaram a emrpesa Cyclic Software oferecendo
suporte ao CVS, e embora a tenham vendido a empresa posteriormente, continuavam
a utilizar o CVS no seu dia a dia. Suas frustrações com os problemas do CVS
foram tais que Jim começou a pensar em uma melhor forma de lidar com
versioanamento.

No início de 2000, a CollabNet contratou Karl para desenvolver um novo
sistema de controle de versão para substituir o CVS na sua suite colaborativa
CollabNet Enterprise Edition (CEE). Jim então convenceu a empresa onde trabalhava,
a Red Hat, a cedê-lo para esse projeto, iniciando assim o desenvolvimento
do Subversion \cite{Sussman}.

O Subversion foi então um dos primeiros sistemas de controle de versão
open source a pensar desde o seu projeto no desenvolvimento colaborativo.

Embora o Subversion tenha sido desenhado para suprir diversas demandas
não satisfeitas pelo CVS, ele ainda era fortemente influenciado pelo seu
antecessor, mantendo por exemplo uma arquitetura centralizada, onde a
figura de um servidor central era necessária durante a operação.

\section{O Linux e o Controle Descentralizado de Versões}

O Linux é de longe um dos projetos de software livre de maior sucesso na
atualidade, e talvez um dos maiores difusores da filosofia open source.

Liberada em 1991 em um post no newsgroup ``comp.os.minix.'' como o resultado
de um hobby, a primeira versão do Linux logo cresceu, ganhando novos
contribuidores e tornando-se um dos principais sistemas operacionais utilizados
em servidores.

\begin{citacao}

Hello everybody out there using minix -\\

I'm doing a (free) operating system (just a hobby, won't be big and professional like gnu) for 386(486) AT clones. This has been brewing since april, and is starting to get ready. I'd like any feedback on things people like/dislike in minix, as my OS resembles it somewhat (same physical layout of the file-system (due to practical reasons) among other things).\\

I've currently ported bash(1.08) and gcc(1.40), and things seem to work. This implies that I'll get something practical within a few months, and I'd like to know what features most people would want. Any suggestions are welcome, but I won't promise I'll implement them :-)\\

Linus (torvalds@kruuna.helsinki.fi)\\

PS. Yes – it's free of any minix code, and it has a multi-threaded fs. It is NOT portable (uses 386 task switching etc), and it probably never will support anything other than AT-harddisks, as that's all I have :-(.\\

\cite{Torvalds}
\end{citacao}

Contando com a ajuda de programadores de todas as partes do mundo, logo
foi necessário utilizar um sistema de controle de versão capaz de gerenciar
toda a complexidade por trás de um desenvolvimento descentralizado.

Para esta tarefa Linus utilizou por vários anos o BitKeeper, ferramenta
proprietária mas que ao contrário do CVS e SVN possui uma arquitetura distribuída,
facilitando o trabalho de colaboradores espalhados mundo afora.

Mas em 2005 o licenciamento do BitKeeper mudou, tornando o seu uso no
projeto Linux inviável. Linus então decidiu criar um sistema de controle
de versão descentralizado tomando como base alguns critérios:

\begin{enumerate}

\item Tomar o CVS como  um exemplo do que não fazer; na dúvida, fazer exatamente
a escolha oposta ao que o CVS fez.

\item Suportar um workflow parecido com o do BitKeeper.

\item Fortíssima segurança contra corrupção, acidenta ou maliciosa

\item Altíssima performance

\end{enumerate}

Com essas premissas em mente Linus começou o desenvolvimento do Git e em menos de duas semanas já tinha implementado várias funcionlaidades. Menos de dois meses depois, o kernel 2.6.12 já era lançado tendo seu código gerenciado pelo próprio Git \cite{Chacon}.


